\documentclass[twoside,11pt]{article}

% Any additional packages needed should be included after jmlr2e.
% Note that jmlr2e.sty includes epsfig, amssymb, natbib and graphicx,
% and defines many common macros, such as 'proof' and 'example'.
%
% It also sets the bibliographystyle to plainnat; for more information on
% natbib citation styles, see the natbib documentation, a copy of which
% is archived at http://www.jmlr.org/format/natbib.pdf

\usepackage{jmlr2e}
\usepackage[acronym,shortcuts,smallcaps,nowarn,nohypertypes={acronym,notation}]{glossaries}

\usepackage{amsmath}
\usepackage{amsfonts}       % blackboard math symbols
\usepackage{amsthm}       % blackboard math symbols
\usepackage{amsfonts}       % blackboard math symbols

\usepackage{bbm}
\usepackage{siunitx}
\usepackage{natbib}


\newcommand{\dataset}{{\cal D}}
\newcommand{\fracpartial}[2]{\frac{\partial #1}{\partial  #2}}

% Heading arguments are {volume}{year}{pages}{date submitted}{date published}{paper id}{author-full-names}

%\jmlrheading{1}{2000}{1-48}{4/00}{10/00}{meila00a}{Marina Meil\u{a} and Michael I. Jordan}

% Short headings should be running head and authors last names

\ShortHeadings{Learning Connect Four}{}
\firstpageno{1}

\begin{document}

\title{Learning Connect Four}

\author{\name Andrea Malleo \email am101@nyu.edu \\}

\maketitle

\begin{abstract}%   <- trailing '%' for backward compatibility of .sty file
%~\cite{chow:68}
In this paper we scale down the framework of AlphaZero,
investigating whether the simpler game of Connect Four can be solved with a commensurate
reduction in model size and compute time. In the end it is clear that the expectations 
for training time reduction were set too high, and more dedicated optimization of certain parameters
would be necessary to better replicate the results. 
\end{abstract}

%\begin{keywords}
  
%\end{keywords}
 
\section{Introduction}
General game playing %\cite{GeneralGamePlay}
 is concerned with unspecialized AI systems
capable of learning to play many different types of games well. Presented here is a reimplementation of the AlphaZero \cite{AlphaZero}method of 
mastering games through reinforcement learning. This method achieved world class performance for 
the games of chess, Go, and shogi.  The sheer magnitude of the resources and time taken to achieve this feat 
is striking, 
but perhaps it just matches the complexity of the problems beings solved. Here, we adapt the framework to Connect Four.
The key questions are how instrinsic to success is this scale, and how general purpose does this method turn out to be.


\section{Background}
A central design choice in AlphaZero 
 is the lack of domain knowledge embedded in the training 
scheme. As such, there are just two game agnostic components to the system: the deep neural network and the
Monte Carlo tree search algorithm. The network $(p,v) = f_{\theta}(s)$
takes the board state over $T$ time steps as input. The outputs are $p$, a probability distribution vector
over the actions to take, and  $v$, the expected outcome of the game from the current state.


Connect Four is particularly simple in comparison to the games on which
AlphaZero is benchmarked. In Connect Four, there are only
as many legal moves as columns, and the maximum of number of turns 
is limited to the amount of squares on the board. 
One measure of the relative complexities of Chess, Go, and Connect Four is with
respect to the state space. Van den Herik \cite{VANDENHERIK2002277}  define state space complexity as the number 
of legal positions reachable from the initial position of the game
and tabulate this metric for various games. The metrics relevant to this project are reproduced in the table
below.

\begin{tabular}{ c c }
        Game & State Space Complexity \\
        \hline
        Connect Four & $10^{14}$ \\
        Chess & $10^{46}$ \\
        Go & $10^{172}$ \\
    \end{tabular}

Unlike Chess and Go, Connect Four has been solved via brute force search \cite{ConnectFourComputer} 
and even via knowledge based methods \cite{ConnectFourKnowledge}. The goal here is to demonstrate
adaptability of this variant of Deep Q learning. The tendency for Deep Q learning to be unstable is 
well known and attributed to correlations between the training samples and the fact that the target 
values are themselves functions of the parameters of the network being optimized. These concerns 
are addressed here as they usually are. Data samples are buffered over a batch of games, essentially 
freezing our Q network ($f_{\theta}$)
while it directs our policy during game play, and are randomly sampled from this batch to produce a
less correlated training data set.


\section{Main Text}


\subsection{Methods}
The data used to train this network is generated via self-play aided by the Monte Carlo tree search.
Every action in every game is decided by constructing a tree of the action space rooted at 
the current state. Repeated over a set of
$n$ trials, the game is advanced by traversing the tree along the path of the highest
value states as determined thus far. If a leaf is reached before the game has ended, a scheme to complete the game must be chosen.
In classic Monte Carlo tree search, moves are chosen at random until the game has a resolution. 
The method used in the AlphaZero paper calls for evaluation of the neural network at this leaf
which will return an expected outcome of the game right away.
The implications of the choice of playout or evaluation
will be discussed in later sections.
Once a score has been obtained, all of the states along the path have their total values updated.
Note that at each level of the tree, the current player switches. Thus, the score of +1 for a win
will be propagated up to every other layer in the tree, while the score of -1 for a loss will be added
to the outcomes for the layers of the tree in between, corresponding to the other losing player. 
A tie yields a 0 outcome. During this  update step, the visit count for each of the states is incremented. 
From total action value and visit count, we can report 
mean values of these nodes/states, which will effect the paths taken at the start of successive trials.

After $n$ simulations have completed, a single action is finally selected proportional to its visit
count, which directly corresponds to its estimated mean value.

Each step produces one training data point. The history of the game over the last $T$ timesteps
is the input image, and the target policy $\pi$ for this state corresponds to the 
visit probabilities of the root node's children. Once the real game finishes, all of these steps
also have an game outcome target, $z$. 
 
The neural network used in this paper is very shallow in comparison to that used in AlphaZero, 
as the state and action space is so reduced. There are just three convolutional layers before
the net splits off into its policy and value heads. Each head has an additional convolutional 
layer followed by a fully connected linear layer. The Adagrad optimizer was chosen with an
initial learning rate on the order of $10^{-3}$. 
The loss function used for parameter optimization is the sum 
over mean squared error and cross entropy losses: \begin{flalign}
    L = (z-v)^2 - \pi^T \log p
\end{flalign}
With such a small network, the real bottleneck in training time was the self-play 
for generating the data. This implementation parallelized gameplay across five python
processes, which approached the maximum capacity for memory on the author's computer.

%For the first three experiments, 
%In effort to reduce the correlation between the training points, were uniformly sampled
%from these as training data. In order to complete 50 rounds of training, between one and two
%hours of time was needed, depending on whether the Monte Carlo simulation used random
%playout or network evaluation to finish the trials.


\subsection{Experiments}
The models were evaluated against two Connect Four players. The first, referred to
as the MCTS player, makes actions using the same MC simulation method detailed above.
The second player, referred to as the random player, takes actions by sampling 
from a uniform probability over all of the columns on the board. 
In the first experiment, three versions of the framework were compared, 
one in which the Q network is used
as intended for outcome evaluation during simulations, a second where the random 
playout method is used instead, and a third combining 
the two approaches, switching to network use halfway through training time.
The results of this stage are contained in Figure 1 and Table 1. All experimental settings
are reported in Table 4 in the Appendix.

\begin{figure}[h]
    \centering
    \includegraphics[scale=0.5]{../NetworkNoNetwork_policy.jpg}\includegraphics[scale=0.5]{../NetworkNoNetwork_val.jpg}
    \caption{Loss over time comparison. What looks like superior convergence for the network design does not yield 
     better validation metrics.}
\end{figure}

\begin{table}[ht]
\caption{Performance ratings are over an average of 500 rounds of play.}
\begin{center}
    \begin{tabular}{||c ||c c c||} 
    \hline
     & Nework & No Network & Half Network \\ [0.5ex] 
    \hline\hline
    Win rate v. MCTS & 0.846 & 0.856 & 0.838\\ 
    \hline
    Draw rate v. MCTS & 0.046 & 0.042 & 0.034 \\
    \hline
    Win rate v. Random & 0.494 & 0.538 & 0.518 \\
    \hline
    Draw rate v. Random & 0.092 & 0.11 & 0.103 \\
    \hline
   \end{tabular}
   \end{center}
\end{table}

There are several takeaways from the data. First is that the convergence of the model
set up as prescribed in AlphaZero looks to be faster than setups using random playout
for state value estimations. However when it comes to performance against the MCTS and Random players,
this approach does no better. In fact in certain instances \footnote{During the presentation I reported
that the MCTS network was increasing probability of a column as the number of plays in that column increased.
This was observed at the time. For the settings of these experiements specifically, all models output almost
uniform probabilities. The version of the model produced in the final experiment does exhibit this strategy.} , the No Network model demonstrated a
discernible strategy whereas the action distributions $p$ observed for the network model were close to equal probability for each column at every step of the game. 
The Network model does learn to assign 0 probability to a column it or the opponent just closed out on the
previous turn, however at future timesteps it often reverts back to uniform distribution.\\ 
An explanation for this trouble is possibly rooted in the loss function (1). The network is provided
target distributions $\pi$ which are the visit probabilities of the direct descendants 
of the root state after a full MC tree search. As noted previously, the use of Q makes for a  non-stationary target $\pi$,
and gradient methods of this approach are not
guaranteed to converge. It is worth noting here that in AlphaGoZero \cite{AlphaGoZero}, the prior work of AlphaZero,
the value estimation is derived by combining network evaluation and rollout outcomes. \\
However, a second related observation to make is that the Random player performs better than the MCTS player.
This speaks to a fundamental flaw in the search algorithm.

Let us more closely examine how exactly 
actions are chosen. From \cite{Supplementary}, an action from state $s$
at time $t$ is selected  as \begin{flalign}
    a &= \arg \max_a (Q(s_t,a) + U(s_t,a))\\
    &\text{Where Q is state value estimation from $f_{\theta}$}\\
    U &= C(s) P(s,a) \frac{\sqrt{N(s)}}{1 +  N(s,a)}\\
    &\text{P(s,a) is the prior probability of taking action a in state s}\\
    &\text{N(s) is parent visit count}\\
    &\text{N(s,a) is node visit count }
\end{flalign}

 The lead suspect is the value for $C$ in the $U$ term, 
influencing which action is chosen, and specifically controlling the rate of exploration.
This is a reported parameter taken directly from the Alpha Zero paper and very likely needs to be tuned 
to Connect Four specifically. Searching for the optimal C(s) is a critical part of future work.\\
Finally, a look at the graph of expected game outcome loss shows little progress over time.
One may wonder whether the relative simplicity of Connect Four hinders the strength 
of state as a signal for expected outcome. The board state before a win can 
be identical to the board state before a loss, and this may happen more often than it does not.


In the baseline experiments, 400 rounds of Monte Carlo trials are used at each 
action step. While this is half of what is used in AlphaZero, the question could
be asked how many are necessary to capture the ideal action to take when 
there are only five choices on a 5x5 board. Performance metrics in Table 2 
show that 25 seems sufficient, and support a rough
scaling heuristic of the number of simulations 
to the square of the number of possible moves,
since on average \cite{thirtyFive} there are 35 possible moves in a state of chess, 
and $35^2 \sim 800$ is the cited number of simulations used in AlphaZero. We confirm
that there is a lower limit before performance degrades in the final column of Table 2. 


\begin{table}[ht]
    \caption{}
\begin{center}
    \begin{tabular}{||c c c c c||} 
    \hline
     & 400 & 100 & 25 & 5 \\ [0.5ex] 
    \hline\hline
    Win rate v. MCTS & 0.838 & 0.834 & 0.86 & 0.646\\ 
    \hline
    Draw rate v. MCTS & 0.034 & 0.038 & 0.034 & 0.07\\
    \hline
    Win rate v. Random &  0.518 & 0.509 & 0.532 &0.47\\
    \hline
    Draw rate v. Random & 0.103 & 0.1046 & 0.106 &0.1\\
    \hline
   \end{tabular}
   \end{center}
\end{table}

% Conjecturing / Data about the number of simulations per step 

So far, the number of timesteps of history provided in a training image has been $T=1$. 
The next set of experiments investigated whether including more history would improve
outcomes. $T=5$ and $T=10$ were chosen. The data suggests that all other paramaters held
constant, no positive change in performance is observed. Therefore the decision to only use
the state of the board at time t is retroactively justified.

\begin{table}[ht]
    \caption{}
\begin{center}
    \begin{tabular}{||c c c c||} 
    \hline
     & T=1 & T=5 & T=10 \\ [0.5ex] 
    \hline\hline
    Win rate v. MCTS & 0.834  & 0.775 & 0.762 \\ 
    \hline
    Draw rate v. MCTS &  0.038  & 0.056 & 0.068 \\
    \hline
    Win rate v. Random &  0.509 & 0.506 & 0.508\\
    \hline
    Draw rate v. Random & 0.1046 & 0.114 & 0.102 \\
    \hline
   \end{tabular}
   \end{center}
\end{table}

With takeaways from each experiment determining the settings for the final model (see Table 4),
the batch size and number of batches was increased substantially, and the network
was trained on a game sized board. 


% Conjecturing / Data about the number of time steps 

% Conjecturing / Data with increasing board size

% graph action distributions as the state changes


\subsection{Conclusion}
AlphaZero was trained for 3.1 million steps over a period of 40 days \cite{AlphaZero}. 
The implementation can be found at https://github.com/Akmalleo3/ConnectFour\_AlphaZero.
Both  \cite{Supplementary} and \cite{refImplementation} 
reference implementations were 
used.

%{\noindent \em Remainder omitted in this sample. See http://www.jmlr.org/papers/ for full paper.}
% use of classical vs. neural network evaluation

% weak state signal 

% double use of q 


% Acknowledgements should go at the end, before appendices and references
%\acks{ }

\vskip 0.2in
\bibliographystyle{plain}
\bibliography{sample}

\appendix
\section*{Appendix A.}
\begin{table}[ht]
    \caption{Hyperparameters}
    \vspace{.3in}
\begin{center}
    \begin{tabular}{||c ||c |c |c |c|c|c|c }
             & Games & Batch Size & T & M & Batches & Board Size &Network\\
             \hline
            Experiment 1&&&&& \\
            1 & 40 & 200 & 1 & 400 & 50 & 5x5 & Yes\\
            2 & 40 & 200 & 1 & 400  & 50 & 5x5 & No\\
            3 & 40 & 200 & 1 & 400 &  50 & 5x5 &Half\\
            \hline
            Experiment 2&&& && \\
            1 & 40 & 200& 1 & 400 &  50 & 5x5 &Half\\
            2 & 40 & 200& 1 & 100 &  50 & 5x5 &Half\\
            3 & 40 & 200& 1 & 25 &  50 & 5x5 &Half\\
            4 & 40 & 200& 1 & 5 &  50 & 5x5 &Half\\
            \hline
            Experiment 3&&&&& \\
            1 & 40 & 200& 1 & 25 &  50 & 5x5 &Half\\
            2 & 40 & 200 & 5 & 25 &  50 & 5x5 &Half\\
            3 & 40 & 200 & 10 & 25 & 50& 5x5 &Half\\
            \hline
            Experiment 4/Best:&&&&& \\
            & 80 & 2048 & 1 & 50 & 150& 6x7 & Half
        \end{tabular}
    \end{center}
\end{table}

\end{document}
