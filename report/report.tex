\documentclass[twoside,11pt]{article}

% Any additional packages needed should be included after jmlr2e.
% Note that jmlr2e.sty includes epsfig, amssymb, natbib and graphicx,
% and defines many common macros, such as 'proof' and 'example'.
%
% It also sets the bibliographystyle to plainnat; for more information on
% natbib citation styles, see the natbib documentation, a copy of which
% is archived at http://www.jmlr.org/format/natbib.pdf

\usepackage{jmlr2e}
\usepackage[acronym,shortcuts,smallcaps,nowarn,nohypertypes={acronym,notation}]{glossaries}

\usepackage{amsmath}
\usepackage{amsfonts}       % blackboard math symbols
\usepackage{amsthm}       % blackboard math symbols
\usepackage{amsfonts}       % blackboard math symbols

\usepackage{bbm}
\usepackage{siunitx}
\usepackage{natbib}


\newcommand{\dataset}{{\cal D}}
\newcommand{\fracpartial}[2]{\frac{\partial #1}{\partial  #2}}

% Heading arguments are {volume}{year}{pages}{date submitted}{date published}{paper id}{author-full-names}

%\jmlrheading{1}{2000}{1-48}{4/00}{10/00}{meila00a}{Marina Meil\u{a} and Michael I. Jordan}

% Short headings should be running head and authors last names

\ShortHeadings{Learning Connect Four}{}
\firstpageno{1}

\begin{document}

\title{Learning Connect Four}

\author{\name Andrea Malleo \email am101@nyu.edu \\}


\begin{abstract}%   <- trailing '%' for backward compatibility of .sty file
%~\cite{chow:68}
This paper scales down the framework of AlphaZero in both complexity and compute power,
investigating whether the simple game of Connect Four can be solved with a commensurate
reduction in model size and training time. 
\end{abstract}

%\begin{keywords}
  
%\end{keywords}
 
\section{Introduction}
Presented here is a reimplementation of the widely known and much awed AlphaZero method of 
mastering games through reinforcement learning. This method is adapted to 
the toy problem of Connect Four. Connect Four is particularly simple. There are only
as many legal moves as columns on the board, and the maximum of number of moves 
is limited to the amount of squares on the board. 

\section{Background}




\section{Main Text}


\subsection{Methods}
A central design choice of AlphaZero is the lack of domain knowledge embedded in the training 
scheme. Its goal is to be a general-purpose framework 
. As such, there are two game agnostic components to the system: the deep neural network and the
Monte Carlo tree search algorithm. The network $(p,v) = f_{\theta}(s)$
takes the board state over $T$ time steps as input and outputs $p$, a probability distribution vector
over the actions to take, and  $v$, the expected outcome of the game from the current state.
The data used to train this network is generated via self-play aided by the MC tree search. 
Every action in every game is decided by constructing a tree of the action space rooted at 
the current state. Over a set of
$n$ trials, the game is advanced by traversing the tree along the path of the highest
value states as determined thus far. If a leaf is reached before the game has ended, i.e. the trial
has reached a state not explored in any of the previous trials, a scheme to complete the game must be chosen.
In classic Monte Carlo tree search, moves are chosen at random until the game has a resolution. 
The method used in the AlphaZero paper calls for evaluation of the neural network in its current state
which will return an expected outcome of the game right away.
 \textbf{The implications of the choice of playout or evaluation
will be discussed in a later section}
Once a score has been obtained, all of the states along the path have their total values updated.
Note that at each level of the tree, the current player switches. Thus, the score of +1 for a win
will be propagated up to every other layer in the tree, while the score of -1 for a loss will be added
to the outcomes for the layers of the tree in between, corresponding to the other losing player. 
A tie yields a 0 outcome. During this  update step, the visit count for each of the states is incremented. 
From total action value and visit count, we can report 
mean values of these nodes/states, which will effect the paths taken at the start of successive trials.

After $n$ simulations have completed, a single action is finally selected proportional to its visit
count, which directly corresponds to its estimated mean value.

Each step produces one training data point. The history of the game over the last $T$ timesteps
is the input image, and the target policy $\pi$ for this state corresponds to the 
visit probabilities of the root node's children. Once the real game finishes, all of these steps
also have an game outcome target, $z$. 
 
The neural network used in this paper is very shallow in comparison to that used in AlphaZero, 
as the state and action space is so reduced. There are just three convolutional layers before
the net splits off itno its policy and value heads. Each head has an additional convolutional 
layer followed by a fully connected linear layer. The Adagrad optimizer was chosen with an
initial learning rate of $10^{-3}$. 
The loss function used for parameter optimization is the sum 
over mean squared error and cross entropy losses: $L = (z-v)^2 - \pi^T \log p$.
With such a small network, the real bottleneck in training time was the self-play 
time for generating the data. This implementation parallelized gameplay across five python
processes, which approached the maximum capacity for memory on the author's computer. Due to
this limitation, only 40 games were played per training round, generating approximately 800 data points. 
In effort to reduce the correlation between the training points, only 200 were uniformly sampled
from these as training data. In order to complete 50 rounds of training, between one and two
hours of time was needed, depending on whether the Monte Carlo simulation used random
playout or network evaluation to finish the trials.


\subsection{Experiments}
The first goal of the project was to get a baseline reimplementation working. 
The models were evaluated against two Connect Four players. The first, referred to
as the MCTS player, makes actions using the same MC simulation method detailed above.
The second player, referred to as the random player, takes actions by sampling 
from a uniform probability over all of the columns on the board. 
Three versions of the framework were compared, one in which the network is used
as intended for outcome evaluation during simulations, a second where the random 
playout method is used to get outcomes for the trials, and a third combining 
the two approaches, switching from no network to network halfway through training time.
The results of this stage are contained in Figure 1 and Table 1.

\begin{figure}[h]
    \centering
    \includegraphics[scale=0.5]{../NetworkNoNetwork_policy.jpg}\includegraphics[scale=0.5]{../NetworkNoNetwork_val.jpg}
    \caption{Loss over time comparison. What looks like superior convergence for the network design does not correspond 
    with better validation metrics.}
\end{figure}

\begin{table}[ht]
\caption{Performance ratings are over an average of 500 rounds of play.}
\begin{center}
    \begin{tabular}{||c ||c c c||} 
    \hline
     & Nework & No Network & Half Network \\ [0.5ex] 
    \hline\hline
    Win rate v. MCTS & 0.846 & 0.856 & 0.838\\ 
    \hline
    Draw rate v. MCTS & 0.046 & 0.042 & 0.034 \\
    \hline
    Win rate v. Random & 0.494 & 0.538 & 0.518 \\
    \hline
    Draw rate v. Random & 0.092 & 0.11 & 0.103 \\
    \hline
   \end{tabular}
   \end{center}
\end{table}

In the baseline experiments, 400 rounds of Monte Carlo trials are used at each 
action step. While this is half of what is used in AlphaZero, the question could
be asked whether 100 rollouts, or even just 25, are sufficient enough to capture
the ideal action to take.

\begin{center}
    \begin{tabular}{||c c c c||} 
    \hline
     & 400 & 100 & 25 \\ [0.5ex] 
    \hline\hline
    Win rate v. MCTS & 0.838 & 0.834 & 0.86 \\ 
    \hline
    Draw rate v. MCTS & 0.034 & 0.038 & 0.034 \\
    \hline
    Win rate v. Random &  0.518 & 0.509 & 0.532 \\
    \hline
    Draw rate v. Random & 0.103 & 0.1046 & 0.106 \\
    \hline
   \end{tabular}
   \end{center}

% Conjecturing / Data about the number of simulations per step 

So far, the number of timesteps of history provided in a training image has been $T=1$. 
The next set of experiments investigated whether including more history would yields
better outcomes. $T=5$ and $T=10$ were chosen. The data suggests that 


\begin{center}
    \begin{tabular}{||c c c c||} 
    \hline
     & T=1 & T=5 & T=10 \\ [0.5ex] 
    \hline\hline
    Win rate v. MCTS & 6 & 87837 & 787 \\ 
    \hline
    Draw rate v. MCTS & 7 & 78 & 5415 \\
    \hline
    Win rate v. Random & 545 & 778 & 7507 \\
    \hline
    Draw rate v. Random & 545 & 18744 & 7560 \\
    \hline
   \end{tabular}
   \end{center}
% Conjecturing / Data about the number of time steps 

% Conjecturing / Data with increasing board size

% graph action distributions as the state changes


\subsection{Discussion}
%{\noindent \em Remainder omitted in this sample. See http://www.jmlr.org/papers/ for full paper.}
% use of classical vs. neural network evaluation

% weak state signal 

% double use of q 


% Acknowledgements should go at the end, before appendices and references
%\acks{ }


\section {References}

\newpage

\appendix
\section*{Appendix A.}

\vskip 0.2in
\bibliographystyle{plain}
\bibliography{sample}
\end{document}




